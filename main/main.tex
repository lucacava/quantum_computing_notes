\documentclass[a4paper]{article}

\usepackage[utf8]{inputenc}
\usepackage{subfiles}
\usepackage{neuralnetwork}
\usepackage{tikz}
\usetikzlibrary{quantikz}
\usepackage{bm} % bold vectors in math mode
\usepackage{amsmath}

\author{Luca Cavallini}
\title{Appunti di quantum computing}
\date{\today}

\begin{document}

\maketitle

\section{Fondamenti}

\section{Gates}

\subsection{NOT gate}

\begin{center}
	\begin{quantikz}	
		& \gate{X} & \qw
	\end{quantikz}
\end{center}

\subsubsection{Definizione matriciale}

\begin{equation}
	X =
	\begin{bmatrix}
		0 & 1 \\
		1 & 0
	\end{bmatrix}
\end{equation}

\subsubsection{Funzionamento}

\begin{equation}
	\begin{aligned}
		& X \ket{0} = \ket{1}  \\  
		& X \ket{1} = \ket{0}  \\  
		& X (\alpha \ket{0} + \beta \ket{1}) = \alpha \ket{1} + \beta \ket{0}
	\end{aligned}
\end{equation}

Dimostriamo l'ultima relazione utilizzando la base $\ket{0} = 	\begin{bmatrix} 1 \\ 0 \end{bmatrix}$, $\ket{1} = \begin{bmatrix} 0 \\ 1 \end{bmatrix}$:

\begin{equation}
	\begin{aligned}
		X (\alpha \ket{0} + \beta \ket{1}) = 
		& \begin{bmatrix}
			0 & 1 \\
			1 & 0
		\end{bmatrix}
		\left(
		\alpha \begin{bmatrix} 1 \\ 0 \end{bmatrix} + \beta \begin{bmatrix} 0 \\ 1 \end{bmatrix}
		\right) =
		\begin{bmatrix}
			0 & 1 \\
			1 & 0
		\end{bmatrix} \begin{bmatrix} \alpha \\ \beta \end{bmatrix} = \\
		& \begin{bmatrix} \beta \\ \alpha \end{bmatrix} = 
		\alpha  \begin{bmatrix} 0 \\ 1 \end{bmatrix} + \beta \begin{bmatrix} 1 \\ 0 \end{bmatrix} =
		\alpha \ket{1} + \beta \ket{0}
	\end{aligned}	
\end{equation}

\subsection{Identity gate}

\begin{equation}
	I =
	\begin{bmatrix}
		1 & 0\\
		0 & 1
	\end{bmatrix}
\end{equation}

\subsection{Hadamard gate}

\begin{center}
	\begin{quantikz}	
		& \gate{H} & \qw
	\end{quantikz}
\end{center}

\subsubsection{Definizione matriciale}

\begin{equation}
	H = \dfrac{1}{\sqrt{2}}
	\begin{bmatrix}
		1&1\\
		1&-1
	\end{bmatrix}
\end{equation}

\subsubsection{Funzionamento}

\begin{equation}
	\ket{0} \mapsto \dfrac{\ket{0} + \ket{1}}{\sqrt{2}}, \ket{1} \mapsto \dfrac{\ket{0} - \ket{1}}{\sqrt{2}}
\end{equation}

Si dimostra utilizzando la base $\ket{0} = 	\begin{bmatrix} 1 \\ 0 \end{bmatrix}$, $\ket{1} = \begin{bmatrix} 0 \\ 1 \end{bmatrix}$:

\begin{equation}
	\begin{aligned}
		& H \ket{0} = \dfrac{1}{\sqrt{2}}
		\begin{bmatrix}
			1 & 1\\
			1 & -1
		\end{bmatrix}	
		\begin{bmatrix}
			1 \\
			0 
		\end{bmatrix} = 
		\dfrac{1}{\sqrt{2}} 
		\begin{bmatrix}
			1 \\
			1
		\end{bmatrix} = \dfrac{1}{\sqrt{2}}
		\begin{bmatrix}
			1\\
			0
		\end{bmatrix} +  \dfrac{1}{\sqrt{2}}
		\begin{bmatrix}
			0\\
			1
		\end{bmatrix} =
		\dfrac{\ket{0} + \ket{1}}{\sqrt{2}}\\
		& H \ket{1} = \dfrac{1}{\sqrt{2}}
		\begin{bmatrix}
			1 & 1\\
			1 & -1
		\end{bmatrix}	
		\begin{bmatrix}
			0 \\
			1 
		\end{bmatrix} = 
		\dfrac{1}{\sqrt{2}} 
		\begin{bmatrix}
			1 \\
			-1
		\end{bmatrix} = \dfrac{1}{\sqrt{2}}
		\begin{bmatrix}
			1\\
			0
		\end{bmatrix} -  \dfrac{1}{\sqrt{2}}
		\begin{bmatrix}
			0\\
			1
		\end{bmatrix} =
		\dfrac{\ket{0} - \ket{1}}{\sqrt{2}} 
	\end{aligned}	
\end{equation}

\end{document}